\myappendix{labnotebook}{Lab Notebook}

\bigskip
\noindent
The instructional laboratory serves various educational purposes which include

\begin{enumerate}
	\item training you to be a practicing scientist.
	\item training you to properly record your experiments in a lab notebook.
	\item covering essential topics such as proper use of equipment, safety, uncertainty in measurements and calculations, and graphical analysis.
	\item providing you with hands-on exercises that help you build conceptual models.
	\item challenging you with open-ended projects where you combine your knowledge and creativity to design an experiment or solve a problem.
	\item giving you additional opportunities to practice applying what you learn in the course.
\end{enumerate}

Your lab notebook should be:

\begin{enumerate}
	\item readable---A person should be able to pick up your lab notebook and completely reproduce the experiment. This requires proper detail, description, and organization. A picture is worth a thousand words, so use sketches regularly! But you don't have to follow proper grammar. (WHAT? That's heresy!) The lab notebook is not like a formal lab report or a journal article. It's a permanent, indisputable record of what you did in the laboratory. As long as your record is unambiguous to the point that someone else can reproduce the experiment, then it's clear enough. However, spelling is important because people hate reading misspelled words and because mispeling werds is imbarasing.
	\item complete---The lab notebook must contain the parts listed below for each experiment and follow proper rules for a lab notebook.
	\item accurate and precise---Experiments should be done with great care. ``\emph{Human error}" is not an acceptable excuse and is NOT the same as ``\emph{experimental error}." If there is human error, then redo the experiment. The quality of an experiment can be judged based on \emph{accuracy} and \emph{precision}.
\end{enumerate}

\subsection*{Elements of recording an experiment}

Each experiment in your lab notebook should have the following elements (taken from Dr. Bowman's guidelines used in CHM 299):

\begin{description}
	\item[Experiment or project number and date] The experiment, project number, or other method of identifying the experiment or test should be recorded here.
	\item[Title] A short descriptive title is very desirable.
	\item[Purpose] The purpose should indicate clearly the reason(s) for the experiment or the test.
	\item[Materials and source] It is important to properly identify the materials being used by part, Vendor (if possible), and any other appropriate description.
	\item[Procedure] Enter the exact procedure as completely and concisely as possible. Remember that enough detail should be included to permit someone else to carry out the experiment and obtain, essentially the same results.  Be as accurate and as quantitative as possible. Include all measurements with clear descriptions of those measurements. If data is stored electronically, then place the data file in an appropriate place on the computer with a descriptive name, and reference the path to the file in your lab notebook. You can even burn the file on a CD and include it with the lab notebook.  Do not hurry!  Take time to do this section right.  Your job may depend on it.
	\item[Drawing] If at all feasible, a well-labeled sketch or drawing (not necessarily to scale) should be prepared showing the equipment or arrangement of equipment used, particularly if special equipment or arrangements are involved.
	\item[Results obtained] Results include your own observations, test results, analyses, etc.  List the results obtained in a form that can be easily understood, e. g., tables, graphs, etc.  Give the details of any tests that were utilized to obtain the reported results. Printouts of graphs should be taped into the lab notebook, and the page number should be referenced. Electronic files of data, graphs, and/or analyses should be saved and referenced.
	\item[Conclusion or summary] Enter a short discussion of the significant results. The elements of invention are novelty, utility and unobviousness. State what you believe to be important about this result and what situations it might apply to. State the general idea to which the results allude.

Record the name of the person or persons who saw you carry out this work or to whom the results were described.  In this connection, it is entirely appropriate and even desirable, to call someone into your laboratory to see the experimental results, and if feasible, to repeat the experiment for someone's benefit in order to demonstrate the new result or product.

\item[Signature and date]

The record should be signed and dated immediately upon completion and no further additions' entered thereafter. Do not back date any record.

\item[Witness and date]

Remember that the witness must be able to testify that he or she read and understood the record in question on the day he or she signed as a witness. Therefore, the witness may be anyone who is able to understand what is written.  Usually this is someone in your research group.

The effective date of the record as it stands is the date of, corroboration, i.e.  the day it is witnessed. Ask someone to read and witness your record as soon as it is completed, but do not ask or allow the witness to backdate his or her signature.  Clearly, to obtain the earliest effective date, it is best to have the witness review your work often.  It is also courteous to reciprocate by reviewing the work of the witness.
\end{description}

\subsection*{Rules for keeping a lab notebook}

\begin{enumerate}
	\item Write in permanent ink.
	\item If you make a mistake, strike through the error once and initial.
	\item Include a table of contents at the beginning of the notebook. After adding a new experiment to your notebook, add an entry to the table of contents page that references the experiment.
	\item Leave no blank space in the notebook. If you wish to start at the top of the next page, then draw a line through the remaining blank space on the previous page.
	\item Use a Table of Contents at the beginning of the lab notebook and enter experiment names and the starting page numbers for those experiments. One should be able to open to the table of contents and see a list of experiments performed and recorded in the lab notebook.
\end{enumerate}
