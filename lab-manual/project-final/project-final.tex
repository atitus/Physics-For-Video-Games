
\myappendix{project-final}{Final Project -- Developing an Original Game}

\longgoal

For this project, you will develop an original game that incorporates moving objects that obey the laws of physics. This includes objects that move with constant velocity or constant acceleration, objects with correct relative motion (such as bullets shot from a moving shooter), projectiles, and objects that explode into pieces (like in the game Asteroids).

\section*{Project Guidelines}

You will develop a game that involves moving objects that obey the laws of physics. It is due on at the beginning of our Final Exam time.\\

\noindent
There are six categories that the project's grade is based on.

\begin{enumerate}
	\item level of difficulty (``How does your game compare to the ones written for this class in terms of difficulty?'' or ``Is the code written for your project similar to A-level exercises in our class in terms of difficulty?'' or ``Does your game include competitive aspects?'')
	\item level of creativity (``Is this game exactly like ones we've done in class or have you reached to do something innovative?'' or ``Did you try to duplicate other popular games?'' or ``Did you add correct physics to a popular game that did not have correct physics?'')
	\item level of independence.  Using resources is good, but you can't copy another program or having someone else write the code for you. Always cite your references, including people who help you. Be sure to cite the page numbers from our class handouts that were used for your project. 
	\item completeness (i.e.  "Does the simulation run?" or  or "Did the program include objects that move with constant velocity?" or "Does the program work as expected?")
	\item quality of documentation ("Did you include relevant references?" or "Did others test your game?")
	\item quality of your presentation (``Did you discuss the purpose of the game?'' or ``Did you discuss the rules of the game?'' or ``Did you discuss the physics principles used in the game?'')
\end{enumerate}

\noindent
You should:

\begin{enumerate}
	\item write a VPython program that includes objects that move.
	\item test your game by having others play it and by asking them for feedback.
	\item answer the questions below in a Word document.
	\item create and deliver a presentation about the game.
\end{enumerate}

\section*{Documentation}

You must write a document in Microsoft Word or pdf format that answers the following questions. The document should be complete; it should have correct grammar; and it should be easy to read and understand. Quality writing and organization is expected. Detail is required. Terse responses will not receive significant credit. 

\begin{enumerate}
	\item What is the purpose of your game?
	\item What are the rules of your game?
	\item How must the game be played (i.e. keystrokes, etc.)?
	\item Is this game like any other game that you've seen or played?
	\item What physics principles were used in your game in order to make it realistic?  Be sure to cite the physics principle(s) (such as projectile motion, constant velocity motion, constant acceleration motion, relative motion, center of mass motion, etc.) and where these principles were discussed in our course handouts. Be sure to reference the page number(s).
	\item Does your game violate any laws of physics? (Note that sometimes this is desirable for playability or artistry.)
	\item Who played your game and what did you learn as a result of their feedback?
	\item How might you improve your game in the future?
	\item What resources did you use to help you in writing the game?  If you used web sites, people (such as Dr. T or Zach), books, or any other resources, you must reference them.
	\item What did you personally get out of this project?
\end{enumerate}

\section*{Presentation}

You will give a seven minute presentation. Five minutes will be for the presentation and two minutes will be for questions. You are expected to create a Keynote, Powerpoint, or pdf document that:

\begin{enumerate}
	\item describes the purpose of the game.
	\item describes the rules of the game.
	\item describes how the game is played.
	\item describes which physics principles were used in the game.
	\item how the game can be improved.
\end{enumerate}

You are also expected to demonstrate your game by playing the game.

Your presentation will be graded on:

\begin{enumerate}
	\item whether others acted interested in your game by giving feedback or asking questions.
	\item whether you spoke clearly and were organized in your presentation.
	\item whether you addressed the items above in your presentation.
	\item whether you demonstrated relative aspects of your game.
	\item whether you were enthusiastic about your game.
\end{enumerate}

