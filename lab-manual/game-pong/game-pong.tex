\lab{GAME -- Pong}

\apparatus
\equip{Computer}
\equip{VPython -- www.vpython.org}

\longgoal

The purpose of this activity is to study the classic arcade game Pong and use VPython to develop both a physical and an unphysical version of the game.

\procedure

\begin{enumerate}

\subsection*{Playing Pong}

\item Go to \url{http://www.ponggame.org/} and play the classic Pong Game. Try both the keyboard and mouse to control the paddle.

\smallframe{Pay attention to the motion of the ball when colliding with the wall and a paddle. \\

1.  Is the collision of the ball and a side wall elastic or inelastic? Explain your answer by referring to observations of the motion of the ball.

\vspace{0.5in}

2.  Are the side walls frictionless or not? Explain your answer by referring to observations of the motion of the ball.


\vspace{0.5in}

3.  Is the collision of the ball and a paddle elastic or inelastic? Explain your answer by referring to observations of the motion of the ball.


\vspace{0.5in}

4.  Is the paddle frictionless or not? Explain your answer by referring to observations of the motion of the ball.

}

\subsection*{Creating a ``bouncing'' puck in a box}

We are going to simulate a puck on an air hockey table that is bouncing around the table. In this simulation, we will assume that the walls and paddle are rigid, frictionless barriers. We will also assume that the puck and barriers make elastic collisions. Thus, the COR is 1 for all collisions. For simplicity, we will draw the puck as a ball and refer to it as a ball.

\item Begin a new program. Import the visual package.

\item Set the size of the scene. You may want to set the height and width of the window in pixels. The example below will set the range to be 10 (meters or whatever unit you wish you use), the width to be 600 pixels, and the height to be 600 pixels.

\begin{myvpython}
scene.range=20
scene.width=600
scene.height=600
\end{myvpython}

\item Create walls at the top, bottom, and sides of the screen.

\item Run your program and verify that you have four walls around the perimeter.

\item Create a ball at the center.

\item Define the initial velocity of the ball (\code{\vpythonline{ball.v=vector(5,8,0)}}), the initial clock reading (\code{t=0}), and the time step (\code{dt=0.01}).

\item Create an infinite while loop.

\item Use \code{rate(100)} to slow down the simulation so that the motion is smooth.

\item Update the position of the ball.

\begin{myvpython}
    ball.pos=ball.pos+ball.v*dt
\end{myvpython}

\item Use an \code{if-elif} statement to check for a collision between the ball and each wall. If there is a collision, change the velocity of the ball in an appropriate way.

\item Run your program and verify that it works properly.

\subsection*{Creating inelastic collisions}

In the real world, a ball would lose energy upon colliding with a rigid barrier. Said another way, the coefficient of restitution is always less than 1. Now we will change the last program by adding friction and a coefficient of restitution.

\item Save your program with a different name so that you don't lose the work that you just did.

\item Make the left wall a ``real wall'' that causes the ball to lose speed upon colliding with the wall. In other words, after colliding with the left wall, the ball's velocity would be reduced by a factor less than 1. You define a variable $COR$ which you can change to be whatever value you want (between 0 and 1).  A COR of 0 means that the ball would stick to the wall. A COR of 1 would be an elastic collision.

\begin{myvpython}
        ball.v.x = -COR*ball.v.x
\end{myvpython}

It helps to use a smaller COR, like 0.5 or less, to see the affect more quickly.

\smallframe{Describe the motion of the ball after a long time. Could we have predicted this given the fact that only one wall results in inelastic collisions?}

Suppose that a wall has a spring in it that ``punches'' the ball during the collision, similar to the bumpers in a pinball machine. Then, you could model this wall by giving it a COR greater than 1.

\item Make one of the walls ``super elastic'' by giving it a COR greater than 1. (This is like the bumper in a pinball machine.)

\item Run your program and observe the effect of the collisions on the motion of the ball.

\subsection*{Adding friction to a collision}

Friction acts parallel to the wall in order to reduce the parallel component of the velocity of the ball.

\item Save your program with a different name so that you do not lose your previous work.

\item Make all collisions elastic collisions (i.e. $COR=1$).

Add friction to the left wall by changing the y-component of the velocity of the ball when it collides with the wall.  Exactly how friction affects the velocity of the ball is a bit complicated. Let's use a simple (albeit unphysical) model that reduces the parallel component of the velocity of the ball by a certain percentage. This is similar to the COR for the perpendicular component of the velocity.

\item When the ball collides with the left wall, change the y-component of the velocity by a factor of 20\% or something like that. (A factor of 1 is no friction and a factor of 0 is maximum friction.)

\begin{myvpython}
        ball.v.y=0.2*ball.v.y
\end{myvpython}

\item Run your program.

\smallframe{Describe the motion of the ball after a long time. Could we have predicted this given the fact that only one wall has friction?}

\subsection*{Making a 1-player Pong game}

\item Create a new VPython program. As always, begin by importing the visual package.

\item It might be nice to set the width and height of the window in pixels. Use the code below to set the range to 20 (m or whatever units you want to imagine), the width to 600 pixels, and the height to 450 pixels. You are welcome to use a larger width and height if you wish.

\begin{myvpython}
scene.range=20
scene.width=600
scene.height=450
\end{myvpython}

\item Create walls for the ceiling and floor. Also create a wall on the left side that has a hole in it that represents a goal. This is similar to what you see in air hockey, for example. You'll need two boxes on the left side, with a space between them for the goal.

\item Create small box as a paddle on the right side. You will eventually use your mouse to move this box up and down.

\item Create a ball and make its initial velocity something like (15,12,0) m/s.

\item Create variables for the clock and time step.

\item Create an infinite while loop. Update the position of the ball. Check for collisions with the walls and paddle and change the velocity accordingly. For now, assume elastic, frictionless collisions. Run your program and verify that everything works as expected.

\item Now we will move the paddle with the mouse. Add the following code inside your infinite while loop. Note that I used the names \code{paddle2}, \code{ceiling}, and \code{floor} for my objects. You will have to changes these to be the same names that you used for your objects. You do not need to retype the comments.

\begin{myvpython}
        #get mouse position and move paddle2
        mouse=scene.mouse.pos
        #check if mouse is within walls and set y-position of paddle to the mouse
        if(mouse.y-paddle2.height/2>floor.y+floor.height/2 and mouse.y+paddle2.height/2<ceiling.pos.y-ceiling.height/2):
            paddle2.pos.y=mouse.y
        else:
            #place paddle at center of mouse is outside the walls
            if(mouse.y-paddle2.height/2<floor.y+floor.height/2):
                paddle2.pos=(18,0,0)
            elif(mouse.y+paddle2.height/2>ceiling.y-floor.height/2):
                paddle2.pos=(18,0,0)
\end{myvpython}

\item Run the program and see if you can control the paddle.

\item Check to see if the ball goes past the paddle or past the left walls. If it does, break out of the loop.

\end{enumerate}

\vspace{5in}

\pagebreak

\analysis

\begin{description}

\item[C] Complete this exercise.

\item[B] Do everything for {\bf C} and the following.

\begin{enumerate}
	\item Make one of the left walls a super-elastic wall with $COR>1$ and one of the left walls an inelastic wall with $COR<1$.
	\item Make half your paddle a super elastic paddle with $COR>1$ and half your paddle an inelastic paddle with $COR<1$. Give each half different colors.
\end{enumerate}

\item[A] Do everything for {\bf B} with the following modifications and additions.

\begin{enumerate}
	\item Place your infinite loop inside another infinite loop. When the ball goes past the paddle or through the goal, increment a score, reset the ball to the middle of the scene, and pause the game and wait for a mouse click or key press. Use the function below to pause the game. Call \code{pause()} when you want the game to pause.

\begin{myvpython}
def pause():
    while True:
        rate(50)
        if scene.mouse.events:
            m = scene.mouse.getevent()
            if m.click == 'left': return
        elif scene.kb.keys:
            k = scene.kb.getkey()
            return
\end{myvpython}

	\item Change your program so that the ball bounces off the paddle in a similar way as the \emph{Pong} game that you played at the beginning of this chapter. Note that the physics is incorrect unless you invent a mechanical device that would cause the ball to bounce in this way.
	
\end{enumerate}




\end{description}

